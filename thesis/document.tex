\documentclass[openany,oneside]{book}

\usepackage{jluthesisUTF8}
%\usepackage{gbt7714}
\usepackage{amsmath}
\usepackage{fontspec}
\usepackage{graphicx}
\usepackage{tikz}
\usepackage{multirow}
\usepackage{lipsum}
\usepackage{graphicx}
\usepackage{tabularx}
\usepackage{listings}
\usepackage{algorithmic}
\usepackage{algorithm}
\usepackage{pgf}
\usepackage{amsthm,amssymb}
%theorem
\theoremstyle{definition}
\newtheorem{definition}{\hspace{2em}定义}[chapter]
\newtheorem{theorem}{\hspace{2em}定理}[chapter]
\newtheorem{lemma}{\hspace{2em}引理}[chapter]
\theoremstyle{definition}
\newtheorem{Proof}{\hspace{sem}证明}[chapter]
\renewcommand{\proofname}{\indent\bf 证明}
\setlength{\parindent}{2em}
%theorem完成
\bibliographystyle{gbt7714-2005}
%\usepackage{cite}


\makeatletter
\newenvironment{breakablealgorithm}
{% \begin{breakablealgorithm}
    \begin{center}
        \refstepcounter{algorithm}% New algorithm
        \hrule height.8pt depth0pt \kern2pt% \@fs@pre for \@fs@ruled
        \renewcommand{\caption}[2][\relax]{% Make a new \caption
            {\raggedright\textbf{\ALG@name~\thealgorithm} ##2\par}%
            \ifx\relax##1\relax % #1 is \relax
            \addcontentsline{loa}{algorithm}{\protect\numberline{\thealgorithm}##2}%
            \else % #1 is not \relax
            \addcontentsline{loa}{algorithm}{\protect\numberline{\thealgorithm}##1}%
            \fi
            \kern2pt\hrule\kern2pt
        }
    }{% \end{breakablealgorithm}
        \kern2pt\hrule\relax% \@fs@post for \@fs@ruled
    \end{center}
}
\makeatother

\usetikzlibrary{arrows}
\usepackage{xeCJK}

\floatname{algorithm}{算法}  
\renewcommand{\algorithmicrequire}{\textbf{输入:}}  
\renewcommand{\algorithmicensure}{\textbf{输出:}} 

%opening
\hypersetup{
    pdftitle    = {your pdf title},
    pdfsubject  = {your pdf subject},
    pdfkeywords = {your pdf keywords},
    pdfauthor   = {your name}
}



\begin{document}

\frontmatter
\sloppy % 解决中英文混排的断行问题,会加入间距,但不会影响断行 ????

%
% 手动在长标题中利用 \par 输入断行,
\ctitle{中文标题}
\etitle{English Title}
                                       % 论文 内容提要
\cthesissummary{
    中文摘要。
}
%                                           % 关键词
\ckeywords{甲, 乙, 丙}

\ethesissummary {
    Engligh abstract goes here.
}

\ekeywords{a, b, c}

\makecover


\pagenumbering{Roman} 
%\pdfbookmark[0]{目~~~~录}{contents}

\tableofcontents
{\xiaosi}
%{\fontsize \fontsize{12.05pt}{14.45pt}\selectfont}
% 清除目录后面空页的页眉和页脚
\clearpage{\pagestyle{empty}\cleardoublepage}

%%% 正文
\mainmatter
\defaultfont                        % 正文使用默认字体,小四,宋体

\chapter{绪论}

	
	
\section{课题的研究背景和现状}
	正则表达式最早由Stephen C.Kleene 在他1956年的论文中提出\cite{Kleen72},在这篇论文中他试图通过正则表达式来刻画神经网络和有限自动机的行为。Ken. Thompson在1968年的论文\cite{Thopson68}中首先对正则表达式的匹配进行了描述。Aho的著作\cite{Aho72a}中对正则表达式和有限自动机的理论进行的综合的叙述。
	正则表达式不但在计算机的基础理论研究中有重要的地位,在实际应用中也有广泛的应用。正则表达式为我们提供了简单、有力而效率高的工具,允许我们用比较短的程序去刻画字符串中复杂的字符串结构。Sedgewick\cite{Sedg92a}中指出,经典正则表达式匹配算法的时间复杂度为\(O\left( mn\right) \),其中\(n\)是字符串的长度而\(m\)是正则表达式模式的长度。
	对于正则表达式而言,我们可以将其看作一个比较简单的程序设计语言,那么我们就可以使用编译原理中所学习的知识去处理正则表达式。我们需要考虑对正则表达式进行词法分析、语法分析然后生成抽象语法树,最后将抽象语法树转化为有限自动机,再和给定的字符串进行匹配。这些经典的步骤的设计与实现在\cite{appel2004modern}中有详细的叙述。\\
	\indent 随着计算机科学的发展以及对文本处理需求的提高,正则表达式本身处理能力也不断地提高。除了经典正则表达式中的并、连结、克莱因星号等操作以外,现有的程序设计语言中为了提高正则表达式的效率,还支持回溯的操作。这同时也对正则表达式的处理技术提出了更高的要求。研究人员考虑并行化处理生成的有限自动机,称为标签化有限自动机("Tagged--Determined-Finite-Automata"),即"TDFA",Aaron Karper\cite{karper2014efficient}在他2014年的硕士毕业论文里给出了一个基于TDFA的新匹配算法,将正则表达式的平均性能提升到了\(O\left( m\right) \)级别,并且给出了在Java程序设计语言中的实现。但是,在他的实现中,我们也可以看出经典的命令式程序设计语言的问题:代码极其冗长,达到了上千行。从软件工程的角度说,这无疑是对于软件可靠性极大的隐患之一。\\
	\indent 和以C语言为代表的命令式程序设计语言相对的,还有以Lisp、 Ocaml、Haskell为代表的函数式程序设计语言。函数式程序设计语言强调数据之间的映射和变换。那么正则表达式的处理过程可以看作从输入代码到目标代码之间的映射。实际上,在Robert Harper的一个非正式的手稿中,他仅用了不到200行代码就实现了经典正则表达式的全部功能。在2010年的ICFP上, 来自德国基尔大学的Fischer等学者基于Haskell程序设计语言提出了函数式高效的正则匹配算法\cite{fischer2010play}。


\section{研究基本内容和意义}
	本文着重考虑软件的高效性和正确性,本文基于2010年的ICFP中Fischer等人的文章,基于Standard ML语言实现一个正则表达式解析引擎,并尝试使用HOL4交互式定理证明器来对程序的正确性进行形式化验证。 Standard ML语言是经过小步语义规范定义的轻量级的函数式程序设计语言,并且在Linux和Windows平台上都有良好的运行支持。HOL4是基于Standard ML程序设计语言开发的、基于高阶逻辑的交互式定理证明器,可以用于证明程序的正确性。和各种各样由人操作保证软件工程的手段相比,由数理逻辑规律所保证的可靠性应当是最强的。

\section{研究内容与论文结构}

	\indent 第一章主要介绍课题的研究背景、研究现状和论文的整体内容和结构\\
	\indent 第二章主要用递归定义的方法定义正则表达式,并借助未经验证过的递归下降正则表达式解析器程序原型来介绍Standard ML程序设计语言。\\
	\indent 第三章将会简要介绍HOL4定理证明,并用形式化的方法给出Fischer[ICFP2010]中算法的实现\\
	\indent 第四章将会介绍测试结果。\\
	\indent 第五章将会进行展望和总结。\\
	\indent 第六章为附录,主要补全对本文展开思路没有影响的证明。

\chapter{插入算法的示例}

可用如下方式插入算法流程:


\begin{breakablealgorithm}
    \caption{ACP 调度器}
    \label{algorithm: ACP}
        \begin{algorithmic}[1]
        \REQUIRE{要调度的作业集合$J$和资源集合$R$}
        \FOR{each r $\in$ R}
        %\STATE{test the c\_exec\_idx and the r\_exec\_idx of r with several benchmarking tools}
        \STATE{使用基准测试工具对$r$进行测试}
        \STATE{获取$r$的$storage_r$和$compute\_power_r$的值}
        \STATE{将$storage_r$和$compute\_power_r$写入到资源限制中}
        \ENDFOR
        \WHILE{true}
        \STATE{根据上次运行的结果,更新任务约束}
        \FOR{each r $\in$ R}
        \IF{r.schedular $\neq$ NULL}
        \IF{r.runningtask $\neq$ NULL}
        \STATE{rt $\leftarrow$ r.runningtask}
        \STATE{st $\leftarrow$ rt.starttime}
        \STATE{et $\leftarrow$ rt.endtime}
        \STATE{pd $\leftarrow$ rt.performancedemand}
        \STATE{sr $\leftarrow$ rt.scheduledresource}
        \STATE{根据(rt, st, et, pd, sr)更新约束条件}
        \STATE{$\slash\slash$ 根据任务的开始时间,结束时间,性能需求和其运行资源的性能添加一条约束}
        \ELSE
        \STATE{r.runningtask $\leftarrow$ NULL}
        \STATE{$\slash\slash$ 将该任务设置为已完成}
        \STATE{将 r.runningtask 从 r.schedular 中移除}
        \STATE{pj $\leftarrow$ r.parentjob}
        \IF{pj.schedular = NULL}
        \STATE{从$J$中删除pj}
        \ENDIF
        \ENDIF
        \ENDIF
        \ENDFOR
        \STATE{新建一个带有资源约束的模型}
        \STATE{将任务约束加入到OPL模型中}
        \STATE{将$J$和$R$作为约束添加到数据中}
        \STATE{求解OPL模型}
        \STATE{按照得出的最优解对任务进行分配}
        \STATE{等待新的事件出现}
        \ENDWHILE
    \end{algorithmic}
\end{breakablealgorithm}

\chapter{基本的数学建模}
\section{本章内容概述}
	本章分为两节,第一节介绍正则表达式的定义。第二节利用一个正则表达式递归下降语法分析器的原型来介绍Standard ML程序设计语言。

\section{正则表达式}
\subsection{正则表达式的递归定义}
	本节主要参考Sipser\cite{Sips05a}计算理论导引中的内容,正则表达式作为和有限自动机相等价的模型,我们不妨从更直观的有限自动机出发来对正则表达式进行定义。考虑篇幅和不影响思路的原因,本节略去了部分部分证明,完整的证明放在第七章附录里。
	\begin{definition}[有限自动机]
		一个有限自动机是一个五元组\(\left( Q,\Sigma,\delta,q_0,F\right) \),其中\\
		\indent \indent1. \(Q\) 是一个有限集被称作状态集。\\
		\indent \indent2. \(\Sigma\) 是一个有限集被称作字母表。\\
		\indent \indent3. \(\delta : Q \times \Sigma \rightarrow Q\) 是转移函数。\\
		\indent \indent4. \(q_0 \in Q\) 是初始状态集。\\
		\indent \indent 5. \(F \subseteq Q\) 是接受状态集。
	\end{definition}
	在给出有限自动机的定义以后,我们继续对有限自动机识别字符串给出一个严格的定义。从直观上讲,只要输入的字符串一步一步地按照状态图规定的状态运行,最终到达接受状态,我们就可以说有限自动机接受了这个输入的字符串。
	\begin{definition}[有限自动机接受语言]
		令 \(M = \left( Q,\Sigma,\delta,q_0,F\right)\) 是一个有限自动机, 令 \( w = w_1 w_2 ... w_n\) 是一个字符串,其中每一个 \(w_i\)都在字母表\(\Sigma\)。 则我们称 \(M\)接受\(w\)当且仅当一个状态序列满足下列条件:\\
		\indent \indent 1. \(r_0 = q_0\)\\
		\indent \indent 2. 对所有的\(i = 0... n -1\)都有\(\delta \left( r_i, w_{i+1}\right) = r_{i+1}\)\\
		\indent \indent 3. \(r_n \in F\)
	\end{definition}
	我们称有限自动机\(M\)识别一个语言\(A\) 当且仅当 \(A = \left\lbrace w | M \mbox{接受} w\right\rbrace \)。\\
	\begin{definition}[正则语言]
		我们称一个语言为正则语言当且仅当存在有限自动机识别这个语言。
	\end{definition}
	
	就像在算术中,我们可以对阿拉伯数字定义 \(+\) \(\times\) 这样的操作,组成 \( 2 \times \left( 3 +2\right) \)这样的算术表达式。在正则语言中,我们也可以对正则语言定义一些操作,我们可以称之为正则操作。
	\begin{definition}
		令 \(A\) 和 \(B\) 是两个正则语言, 我们定义并、连结和克莱因星号操作如下:\\
		\indent \indent 1.并: \(A \cup B = \left\lbrace x|x \in A \mbox{或} x \in B \right\rbrace \)\\
		\indent \indent 2.连结: \(A \circ B = \left\lbrace xy| x \in A \mbox{且} y\in B \right\rbrace\)\\
		\indent \indent 3.克莱因星号 \(A^* = \left\lbrace x_1 x_2 ...x_k | k \ge 0 \mbox{并且所有的} x_i \in  A \right\rbrace\)\\
		\indent \indent 在优先级上,克莱因星号是最高级的,连结其次,并的优先级是最低的
	\end{definition}
	我们可以证明下面的内容:
	\begin{theorem}
		所有的正则操作都是在正则语言里封闭的,也就是说,所有的正则语言经过正则操作以后都可以被有限自动机所识别
	\end{theorem}

	
	\indent 继续刚才算术中的例子,正如我们可以用算术操作符组合出算术表达式,我们也可以将正则语言用正则操作组合出正则表达式。
	我们不妨递归地进行定义。
	\begin{definition}[正则表达式的递归定义]
		我们称 \(R\) 是一个正则表达式, 如果\(R\)是 \\
		\indent \indent 1. \(a\) 对字母表\(\Sigma\) 中的任意字母 a\\
		\indent \indent 2. \(\epsilon\)\\
		\indent \indent 3. $\emptyset$\\
		\indent \indent 4. \(\left(  R_1 \cup R_2 \right) ,\mbox{其中}R_1\mbox{和}R_2 \mbox{都是正则表达式}  \)\\
		\indent \indent 5. \( \left( R_1 \circ R_2 \right),\mbox{其中}R_1\mbox{和}R_2\mbox{都是正则表达式}\)\\
		\indent \indent 6.\(\left( R_1^* \right) , \mbox{其中R}_1\mbox{是一个正则表达式}\)\\
		\indent 
	\end{definition}
	其中,表达式\(\epsilon\)代表语言的是一个空字符串,\(emptyset\)代表的是空集,即其中不包含任何字符串。\\
	\indent 我们也可以证明:
	\begin{theorem}
		这个正则表达式的定义和有限自动机是等价的
	\end{theorem}

	
\subsection{用上下文无关文法描述正则表达式}
	基于定义3.5,我们可以使用上下文无关文法递归地描述正则表达式。但是在此之前,我们需要修改一下正则表达式约定的符号。由于计算机命令行程序难以处理"\(\left(\right)\)",并且没有"\(\cup\)"符号和\(\circ\),我们约定:\(\mbox{"}\left(\mbox{"和"}\right)\mbox{"}\)修改为\(\mbox{"}\left[  \mbox{"和"}\right]\mbox{"}\)。并且,我们用"\(+\)"代表"并"操作,即"\(\cup\)",用"\(r_1 r_2\)",也就是两个正则表达式直接连写来代表连结操作。
	
	现在,我们可以把定义3.5改写成如下的形式
	\begin{equation}
		r::= \quad \emptyset \quad| \quad\epsilon\quad |\quad a \quad  |\quad r_1 r_2\quad |\quad r_1 + r_2\quad | \quad r^*
	\end{equation}
	
	我们首先从优先级最低的并操作开始,我们约定,上下文无关文法的起始符是rexp,那么,我们可以得到:
	\begin{equation}
		rexp \quad::=\quad rterm\quad  | \quad rterm + rexp
	\end{equation}
	\indent 起始符的推导有两种情况,如果有并操作的话,这种情况可以分解为两个子表达式,另外一种情况是直接推导出下一个子表达式。对于在并操作下的情况,为了避免在后面的递归下降分析出现左递归的情况, 我们规定 "$+$"符号左边是新的非终结符。如果后续的表达式中还会出现"$+$"操作符,那继续展开右边的 $rexp$ 即可。\\
	\indent 对于优先级居中,而且也是双目操作的连结操作,我们可以像连结操作一样处理。这是因为:根据递归定义,我们可以把优先级更高、暂时不需要展开的正则表达式视作一个整体。在这种情况下,有更高等级的双目操作符可以抽象为同一个情况。在这种情况下,我们有:
	\begin{equation}
		rterm\quad::=\quad rfac\quad | \quad rfac\quad rterm
	\end{equation}
	\indent 对于优先级最高,而且是单目操作符的克莱因星号,用上下文无关文法描述是比较容易的:
	\begin{equation}
		rfac\quad::=\quad ratom \quad | \quad ratom^* 
	\end{equation}
	\indent 根据递归定义,最低级的非终结$ratom$对应两种情况,第一种情况是直接推导出非终结符,第二种情况是推导出括号下的作为一个整体的正则表达式。
	\begin{equation}
		ratom\quad::=\quad @ \quad | \quad ! \quad |\quad a \quad | \quad[\quad rexp \quad]
	\end{equation}
	\indent 其中$@$对应的是空集, $!$ 对应的是空串, $a$对应的是字母表中任意一个字母, $\quad[\quad rexp \quad]$对应的是可能存在是括号内作为一个整体的正则表达式。 \\
	\indent 综上所述,我们得到了正则表达式的上下文无关文法的定义:
	\begin{equation}\label{cfg}
		\begin{split}
			rexp \quad::&=\quad rterm\quad  | \quad rterm + rexp \hfill  \\
			rterm\quad::&=\quad rfac\quad | \quad rfac\quad rterm \hfill  \\
			rfac\quad::&=\quad ratom \quad | \quad ratom^* \hfill \\
			ratom\quad::&=\quad @ \quad | \quad ! \quad |\quad a \quad | \quad[\quad rexp \quad] \hfill 
		\end{split}
	\end{equation}
	\indent 从\(\left( 3-6\right) \)中我们不难看出,每一个非终结符都能推导出下一个非终结符,而最后一个非终结符$ratom$可以推导出终结符,这样的话,我们可以说\ref{cfg}中的上下文无关文法是可达的。
	
	

	
	
\section{Standard ML 语言简介}
	在这一节中,我们主要参考的是Standard ML语言规约文档\cite{harper1986standard}。这是在1997年基于小步语义所指定出来的,后来standard ML程序设计语言所有的实现都是基于这个规约文档。 


\subsection{Standard ML 语言的实现和运行环境}
	Standard ML语言有两种运行模式,一种是由编译器直接编译产生可执行代码,另外一种运行模式是函数式程序设计语言所特有的“读入-计算-输出”循环(英语:"Reading-Evaluating-Print" Loop),也被称作为交互式顶层构件(英语:"Interactive Toplevel"),是一个比较简单的,和程序的解释器、编译器进行交互的运行环境\cite{abelson1996structure}。
	

三级标题

%最后设置格式,插入参考文献。
\defaultfont
\bibliographystyle{gbt7714-2005}
\clearpage
\phantomsection
\addcontentsline{toc}{chapter}{参考文献}
\bibliography{document}
%插入致谢
\chapter*{致 \qquad 谢}
\addcontentsline{toc}{chapter}{致谢}
\thispagestyle{empty}
本科生涯看似漫长却又一晃而过,回首走过的岁月,我感慨良多。从最初的论文选题、思路梳理到研讨交流、反复修改直至最终完稿,都离不开老师、同学和亲人们的支持和无私帮助,在此我要向他们表达我最诚挚的谢意。


...

求学生涯暂告段落,但求知之路却永无止境。我将倍加珍惜大学生活给予我的珍贵财富,不忘初心,砥砺前行!
\chapter*{本科期间发表论文和科研情况}
\addcontentsline{toc}{chapter}{本科期间发表论文和科研情况}
\thispagestyle{empty}

...

\end{document}
