\documentclass[openany,oneside]{book}

\usepackage{jluthesisUTF8}
%\usepackage{gbt7714}
\usepackage{amsmath}
\usepackage{fontspec}
\usepackage{graphicx}
\usepackage{tikz}
\usepackage{multirow}
\usepackage{lipsum}
\usepackage{graphicx}
\usepackage{tabularx}
%代码块
\usepackage{listings}
\setcounter{secnumdepth}{4}
\lstset{
	language = ml,
	backgroundcolor = \color{yellow!10},    % 背景色:淡黄
	basicstyle = \small\ttfamily,           % 基本样式 + 小号字体
	rulesepcolor= \color{gray},             % 代码块边框颜色
	breaklines = true,                  % 代码过长则换行
	numbers = left,                     % 行号在左侧显示
	numberstyle = \small,               % 行号字体
	keywordstyle = \color{black},            % 关键字颜色
	commentstyle =\color{blue!100},        % 注释颜色
	stringstyle = \color{red!100},          % 字符串颜色
	frame = shadowbox,                  % 用(带影子效果)方框框住代码块
	showspaces = false,                 % 不显示空格
	columns = fixed,                    % 字间距固定
	%escapeinside={<@}{@>}              % 特殊自定分隔符:<@可以自己加颜色@>
	morekeywords = {as},                % 自加新的关键字(必须前后都是空格)      % 删除内定关键字;删除错误标记的关键字用deletekeywords删!
}
%代码块结束
\usepackage{algorithmic}
\usepackage{algorithm}
\usepackage{pgf}
\usepackage{amsthm,amssymb}
%theorem
\theoremstyle{definition}
\newtheorem{definition}{\hspace{2em}定义}[chapter]
\newtheorem{theorem}{\hspace{2em}定理}[chapter]
\newtheorem{lemma}{\hspace{2em}引理}[chapter]
\theoremstyle{definition}
\newtheorem{Proof}{\hspace{sem}证明}[chapter]
\renewcommand{\proofname}{\indent\bf 证明}
\setlength{\parindent}{2em}
%theorem完成
\bibliographystyle{gbt7714-2005}
%\usepackage{cite}


\makeatletter
\newenvironment{breakablealgorithm}
{% \begin{breakablealgorithm}
    \begin{center}
        \refstepcounter{algorithm}% New algorithm
        \hrule height.8pt depth0pt \kern2pt% \@fs@pre for \@fs@ruled
        \renewcommand{\caption}[2][\relax]{% Make a new \caption
            {\raggedright\textbf{\ALG@name~\thealgorithm} ##2\par}%
            \ifx\relax##1\relax % #1 is \relax
            \addcontentsline{loa}{algorithm}{\protect\numberline{\thealgorithm}##2}%
            \else % #1 is not \relax
            \addcontentsline{loa}{algorithm}{\protect\numberline{\thealgorithm}##1}%
            \fi
            \kern2pt\hrule\kern2pt
        }
    }{% \end{breakablealgorithm}
        \kern2pt\hrule\relax% \@fs@post for \@fs@ruled
    \end{center}
}
\makeatother

\usetikzlibrary{arrows}
\usepackage{xeCJK}

\floatname{algorithm}{算法}  
\renewcommand{\algorithmicrequire}{\textbf{输入:}}  
\renewcommand{\algorithmicensure}{\textbf{输出:}} 

%opening
\hypersetup{
    pdftitle    = {your pdf title},
    pdfsubject  = {your pdf subject},
    pdfkeywords = {your pdf keywords},
    pdfauthor   = {your name}
}



\begin{document}

\frontmatter
\sloppy % 解决中英文混排的断行问题,会加入间距,但不会影响断行 ????

\input{preface/cover}

\pagenumbering{Roman} 
%\pdfbookmark[0]{目~~~~录}{contents}

\tableofcontents
{\xiaosi}
%{\fontsize \fontsize{12.05pt}{14.45pt}\selectfont}
% 清除目录后面空页的页眉和页脚
\clearpage{\pagestyle{empty}\cleardoublepage}

%%% 正文
\mainmatter
\defaultfont                        % 正文使用默认字体,小四,宋体

\chapter{绪论}

	
	
\section{课题的研究背景和现状}
	正则表达式最早由Stephen C.Kleene 在他1956年的论文中提出\cite{Kleen72},在这篇论文中他试图通过正则表达式来刻画神经网络和有限自动机的行为。Ken. Thompson在1968年的论文\cite{thompson1968programming}中首先对正则表达式的匹配进行了描述。Aho的著作\cite{Aho72a}中对正则表达式和有限自动机的理论进行的综合的叙述。
	正则表达式不但在计算机的基础理论研究中有重要的地位,在实际应用中也有广泛的应用。正则表达式为我们提供了简单、有力而效率高的工具,允许我们用比较短的程序去刻画字符串中复杂的字符串结构。Sedgewick\cite{Sedg92a}中指出,经典正则表达式匹配算法的时间复杂度为\(O\left( mn\right) \),其中\(n\)是字符串的长度而\(m\)是正则表达式模式的长度。
	对于正则表达式而言,我们可以将其看作一个比较简单的程序设计语言,那么我们就可以使用编译原理中所学习的知识去处理正则表达式。我们需要考虑对正则表达式进行词法分析、语法分析然后生成抽象语法树,最后将抽象语法树转化为有限自动机,再和给定的字符串进行匹配。这些经典的步骤的设计与实现在\cite{appel2004modern}中有详细的叙述。\\
	\indent 随着计算机科学的发展以及对文本处理需求的提高,正则表达式本身处理能力也不断地提高。除了经典正则表达式中的并、连结、克莱因星号等操作以外,现有的程序设计语言中为了提高正则表达式的效率,还支持回溯的操作。这同时也对正则表达式的处理技术提出了更高的要求。研究人员考虑并行化处理生成的有限自动机,称为标签化有限自动机("Tagged--Determined-Finite-Automata"),即"TDFA",Aaron Karper\cite{karper2014efficient}在他2014年的硕士毕业论文里给出了一个基于TDFA的新匹配算法,将正则表达式的平均性能提升到了\(O\left( m\right) \)级别,并且给出了在Java程序设计语言中的实现。但是,在他的实现中,我们也可以看出经典的命令式程序设计语言的问题:代码极其冗长,达到了上千行。从软件工程的角度说,这无疑是对于软件可靠性极大的隐患之一。\\
	\indent 和以C语言为代表的命令式程序设计语言相对的,还有以Lisp、 Ocaml、Haskell为代表的函数式程序设计语言。函数式程序设计语言强调数据之间的映射和变换。那么正则表达式的处理过程可以看作从输入代码到目标代码之间的映射。实际上,在Robert Harper的一个1999年的论文中\cite{harper1999proof}中实现的正则匹配算法仅了几十行代码,就实现了正则表达式的匹配功能。在2010年的ICFP上, 来自德国基尔大学的Fischer等学者基于Haskell程序设计语言也提出了高效的基于函数式编程的正则匹配算法\cite{fischer2010play}。时间复杂度为\(O\left( mlogn\right) \)


\section{研究基本内容和意义}
	本文着重考虑软件的正确性,本文利用递归下降语法制导分析处理正则表达式,并基于Robert Harper 论文\cite{harper1999proof}中提出的匹配算法,使用Standard ML语言和SML/NJ编译器,实现一个正则表达式解析程序。
	
	传统的软件测试总是希望尽可能多地去覆盖软件运行的各种情况,从而发现其中的各种错误。而证明导向的程序调试是希望利用数学证明来保证程序的正确性,又由于函数式的程序和数学表达式比较接近,这样的话就能以比较高效和可靠的办法保证程序的正确性。

\section{研究内容与论文结构}

	\indent 第一章主要介绍课题的研究背景、研究现状和论文的整体内容和结构\\
	\indent 第二章是论文中的基本概念:主要包括:表达式简介和Standard ML程序设计语言基础,词法分析和语法分析简介。\\
	\indent 第三章是基于函数式编程的需求分析和模块设计\\
	\indent 第四章是词法分析程序和语法分析程序的实现\\
	\indent 第六章是基于高阶函数的字符串和抽象语法树匹配的算法、证明和实现。\\
	\indent 第七章是展望和反思。\\
	\indent 第八章为附录,主要补全对本文展开思路没有影响的证明以及附上完整的程序  。



\chapter{基本概念和技术简介}
\section{本章内容概述}
	本章主要介绍论文中一些重要的基本概念。本章分为四节,第一节介绍正则表达式的定义。第二节介绍Standard ML程序设计语言,第三节介绍词法分析的基本概念,第四节介绍递归下降语法制导分析的基本概念。
\section{正则表达式}
\subsection{正则表达式的递归定义}
	本节主要参考Sipser\cite{Sips05a}计算理论导引中的内容,正则表达式作为和有限自动机相等价的模型,我们不妨从更直观的有限自动机出发来对正则表达式进行定义。考虑篇幅和不影响思路的原因,本节略去了证明,完整的证明放在第七章附录里。
	\begin{definition}[有限自动机]
		一个有限自动机是一个五元组\(\left( Q,\Sigma,\delta,q_0,F\right) \),其中\\
		\indent \indent1. \(Q\) 是一个有限集被称作状态集。\\
		\indent \indent2. \(\Sigma\) 是一个有限集被称作字母表。\\
		\indent \indent3. \(\delta : Q \times \Sigma \rightarrow Q\) 是转移函数。\\
		\indent \indent4. \(q_0 \in Q\) 是初始状态集。\\
		\indent \indent 5. \(F \subseteq Q\) 是接受状态集。
	\end{definition}
	在给出有限自动机的定义以后,我们继续对有限自动机识别字符串给出一个严格的定义。从直观上讲,只要输入的字符串一步一步地按照状态图规定的状态运行,最终到达接受状态,我们就可以说有限自动机接受了这个输入的字符串。
	\begin{definition}[有限自动机接受语言]
		令 \(M = \left( Q,\Sigma,\delta,q_0,F\right)\) 是一个有限自动机, 令 \( w = w_1 w_2 ... w_n\) 是一个字符串,其中每一个 \(w_i\)都在字母表\(\Sigma\)。 则我们称 \(M\)接受\(w\)当且仅当一个状态序列满足下列条件:\\
		\indent \indent 1. \(r_0 = q_0\)\\
		\indent \indent 2. 对所有的\(i = 0... n -1\)都有\(\delta \left( r_i, w_{i+1}\right) = r_{i+1}\)\\
		\indent \indent 3. \(r_n \in F\)
	\end{definition}
	我们称有限自动机\(M\)识别一个语言\(A\) 当且仅当 \(A = \left\lbrace w | M \mbox{接受} w\right\rbrace \)。\\
	\begin{definition}[正则语言]
		我们称一个语言为正则语言当且仅当存在有限自动机识别这个语言。
	\end{definition}
	
	就像在算术中,我们可以对阿拉伯数字定义 \(+\) \(\times\) 这样的操作,组成 \( 2 \times \left( 3 +2\right) \)这样的算术表达式。在正则语言中,我们也可以对正则语言定义一些操作,我们可以称之为正则操作。
	\begin{definition}
		令 \(A\) 和 \(B\) 是两个正则语言, 我们定义并、连结和克莱因闭包操作如下:\\
		\indent \indent 1.并: \(A \cup B = \left\lbrace x|x \in A \mbox{或} x \in B \right\rbrace \)\\
		\indent \indent 2.连结: \(A \circ B = \left\lbrace xy| x \in A \mbox{且} y\in B \right\rbrace\)\\
		\indent \indent 3.克莱因星号 \(A^* = \left\lbrace x_1 x_2 ...x_k | k \ge 0 \mbox{并且所有的} x_i \in  A \right\rbrace\)\\
		\indent \indent 在优先级上,克莱因闭包是最高级的,连结其次,并的优先级是最低的
	\end{definition}
	我们可以证明下面的内容:
	\begin{theorem}
		所有的正则操作都是在正则语言里封闭的,也就是说,所有的正则语言经过正则操作以后都可以被有限自动机所识别
	\end{theorem}

	
	\indent 继续刚才算术中的例子,正如我们可以用算术操作符组合出算术表达式,我们也可以将正则语言用正则操作组合出正则表达式。
	我们不妨递归地进行定义。
	\begin{definition}[正则表达式的递归定义]
		我们称 \(R\) 是一个正则表达式, 如果\(R\)是 \\
		\indent \indent 1. \(a\) 对字母表\(\Sigma\) 中的任意字母 a\\
		\indent \indent 2. \(\epsilon\)\\
		\indent \indent 3. $\emptyset$\\
		\indent \indent 4. \(\left(  R_1 \cup R_2 \right) ,\mbox{其中}R_1\mbox{和}R_2 \mbox{都是正则表达式}  \)\\
		\indent \indent 5. \( \left( R_1 \circ R_2 \right),\mbox{其中}R_1\mbox{和}R_2\mbox{都是正则表达式}\)\\
		\indent \indent 6.\(\left( R_1^* \right) , \mbox{其中R}_1\mbox{是一个正则表达式}\)\\
		\indent 
	\end{definition}
	其中,表达式\(\epsilon\)代表语言的是一个空字符串,\(emptyset\)代表的是空集,即其中不包含任何字符串。\\
	\indent 我们也可以证明:
	\begin{theorem}
		这个正则表达式的定义和有限自动机是等价的
	\end{theorem}

	

	


\section{Standard ML 语言简介}
	Standard ML语言最早是为了爱丁堡大学的可计算函数逻辑定理证明器(LCF)而设计出来的。后来逐渐用于编译器设计、交互式验证等领域。而在这一节中,我们主要参考的是Standard ML语言规约文档\cite{harper1986standard}。是由卡耐基梅隆大学的Robert Harper等学者基于小步语义所制定出来的,这个文档严格地刻画了Standard ML语言的行为,后来Standard ML程序设计语言所有的编译器和运行环境的实现都是基于这个规约文档。 


\subsection{Standard ML 语言的实现和运行环境}
	Standard ML语言有两种运行模式,一种是由编译器直接编译产生可执行代码,另外一种运行模式是函数式程序设计语言所特有的“读入-计算-输出”循环(英语:"Reading-Evaluating-Print" Loop),也被称作为交互式顶层构件(英语:"Interactive Toplevel"),是一个比较简单的,和程序的解释器、编译器进行交互的运行环境\cite{abelson1996structure}。\\
	\indent Standard ML语言的实现有很多种,其中本论文主要涉及两个实现:第一个是SML/New Jersey,由美国的普林斯顿大学开发维护,被广泛应用于程序设计的课堂教学,以REPL环境为主,在本文的所有代码都可以在这个编译器下直接运行;第二个则是Poly/ML编译器,这个编译器用于实现两个大型定理证明器项目———由慕尼黑工业大学和剑桥大学支持开发的Isabelle/HOL和剑桥大学和查尔姆斯理工学院支持的HOL4,这个编译器既支持REPL环境又可以编译出一个可执行文件。
\subsection{Standard ML语言中的类型}
	Standard ML是静态强类型语言,也就是说每一个变量都必须有对应的类型,否则程序不可能通过编译器的类型检查,这就要求我们无论是在声明变量还是在编写函数的时候都要注意变量和参数的类型。下面为我们可以看几个在REPL环境中的例子。例如:
	\begin{lstlisting}
		val pi = 3.14;
		> val pi = 3.14 : real
	\end{lstlisting}
	这就说明变量\(pi\)的类型是\(real\)。
	和其他的程序设计语言类似,Standard ML 语言也有诸如字符\(char\)、字符串、(、布尔值("bool")等类型。这些类型之间也可以相互转换。例如:
	\begin{lstlisting}
		val what = true;
		> val what = true : bool
		(*调用类型转换函数*)
		Bool.toString(true)
		> val it = "true" : string
		(*it 是返回值的引用*)
	\end{lstlisting}
	刚才的程序片段中出现了将\(bool\)类型转化为\(string\)类型的函数,实际上, Standard ML中的函数可以看作从从类型到类型的映射,比如刚才的\(toString\)函数,就可以当作\(bool\)到\(string\)的一个映射,在Standard ML语言中表示为:
	\begin{lstlisting}
		> val toString : bool -> string
	\end{lstlisting}
	除了程序语言规范中给出的类型以外,我们还可以在Standard ML语言中自行定义数据类型。\cite{paulson1996ml-124}这里不妨以本文的研究对象正则表达式作为一个例子,根据上文对正则表达式的定义,我们可以定义一个正则表达式的数据类型\(Regex\)
	\begin{lstlisting}
		datatype regexp = None
		| Empty
		| Char of char
		| AnyChar
		| Or of regexp * regexp
		| Concat of regexp * regexp
		| Star of regexp;
	\end{lstlisting}
	上面这个声明了八个对象,数据类型\(regexp\),和这个数据类型的七个构造子\(Empty\)、\(Char\)、\(AnyChar\)、\(Or\)、\(Concat\)和\(Star\),以及每个构造子所对应的类型。例如,\(Or\)构造子必须由两个\(regex\)类型的元素作为参数去构造。每一个数据类型的变量包含且仅包含其构造子所构造的值。
\subsection{Standard ML 语言函数式特性概览}
	函数式程序设计中以数据的变换取代了状态,因而在函数式编程中程序没有数组、循环和赋值语句。相应的,作为支持函数式编程的语言,Standard ML有一系列的特性来解决问题。\\
	\subsubsection*{元组}
	\indent 元组(tuple) 是Standard ML中最简单的数据结构,例如
	\begin{lstlisting}
		(*两个int类型变量组成的元组*)
		val pair : int * int = (2,3)
	\end{lstlisting}
	其中,元组的类型可以不同,元组中还可以嵌套元组。\\
	\indent Standard ML只能支持函数返回一个参数,但是基于元组,我们可以让函数返回多个参数。
	\subsubsection*{表}
	\indent 表(list)。数据集可以组织成为下面的表这样的形式来处理。
	\begin{lstlisting}
		[a,b,c,d,e]
	\end{lstlisting}
	表支持顺序访问,从左到右地进行扫描。和Lisp系语言一样, Standard ML也可以对表进行处理。在本文中,对\(list\)最重要的操作符是::操作符:
	\begin{lstlisting}
		val (op ::) : typ * typ list -> typ list		
	\end{lstlisting}
	其中, \(op\)的意思是这是一个函数。::操作符的意义是,给定一个表\(it\),\(h::t\)会把这个表分为两个部分,表头第一个元素$h$和剩下的部分$t$。	
	\indent 函数(function), Standard ML中的表达式主要由函数调用构成的。函数的参数可以是任何类型,包括函数本身。这部分特性在介绍CPS和高阶函数时会进行更详细的说明。Standard ML语言的垃圾收集功能\cite{paulson1996ml-5}支持了这个特性。\\
	\subsubsection*{模式匹配}
	\indent 模式匹配(pattern matching),这是Standard ML语言中非常重要的一个特性,这个特性一般用于多个分支的时候,跳转到不同的分支情况,这个特性可以代替命令式程序语言中的case语句。模式匹配在Standard ML中有两种形式。\\
	\indent 第一种形式被称作定义函数表达式,一般的形式如下:
	\begin{lstlisting}
		fun pat_1 => exp_1
			|...
			|pat_n => pat_n
	\end{lstlisting}
	其中每一个\(pat_i\)是一个模式而每一个\(exp_i\)是一个表达式。其中的每一个元素 \( pat => exp\),被称作为一个定义,或者是一条规则。所有的规则组合起来被称作一个匹配。\\
	\indent 匹配对模式和表达式的类型都有严格的要求,模式要求每一个规则的模式都是相同的类型\(typ_1\)而每一个表达式的类型也是相同的\(typ_2\)。这样才能建立起来映射\(typ_1 \rightarrow typ_2\)。
	
	这里我们可以给模式匹配举一个简单的例子:
	\begin{lstlisting}
		(*字符串元素计数*)
		fun length []= 0
			| length (x::xs) =1 + length xs
	\end{lstlisting}
	
	此外,模式匹配还有另外一种格式,如果对一个表达式\(exp\)的各种情况进行匹配,那么也考虑下面这种形式:
	\begin{lstlisting}
		case exp
		of pat1 => exp1
		| ...
		| patn => expn
		(*是下面这种情况的简写*)
		(fn pat1 => exp1
		| ...
		| patn => expn)
		exp.er
	\end{lstlisting}
	程序会首先计算\(exp\)的值,然后在\(case\)\(of\)语句下的各种情况进行匹配。
	\subsubsection*{高阶函数}
	高阶函数(Higher Order Function)。高阶函数是函数式程序设计中重要的一种特性。函数本身也是可计算的值,也能作为参数传入另外一个函数。更直接地说,高阶函数是一种操作于另外一个或几个函数之上的函数,这里不妨举一个例子:对函数\(f\left(x_1,x_2...x_n \right) \),当这个函数被另外一个函数\(g\)作用时,原来的\(f\)函数就变换为\(f\left[ g\left( x_1\right) ,g\left( x_2\right) ....g\left( x_n\right)\right]    \)。
	\subsection{Standard ML中的访问控制}
	众所周知现在流行的面向对象程序设计的三大特性为封装、继承和多态,其中封装这个特性主要是为了保证对程序中一些数据的访问控制。事实上,虽然Standard ML语言作为一个函数式语言不支持面向对象的特性,但是Standard ML也自有一套机制用来保证对程序中数据成员的访问控制。这就是模块(module)系统\cite{harper1989introduction}。
	
	
	模块系统主要分为两个部分,签名(signature)和结构体(structure)。签名是对结构体的抽象和描述,主要规定了一个模块中的数据类型,以及外界可以访问的绑定数据和成员函数。结构体是对一个签名的具体实现,主要是对签名里声明的函数的实现。一个结构体也可以声明签名没有声明过的函数,这些函数外界都无法进行访问。和面向对象程序设计中的组合类似,签名中也可以声明其他结构体后使用结构体中可以访问的成员函数。 
	
	接下来简要介绍一下签名和结构体的语法。首先从签名开始,签名可以包括四个部分: 类型声明和数据类型声明、对异常的规定和值的声明,其中值的声明包括函数的声明和绑定值的声明,下面的代码便是一个例子:
	\begin{lstlisting}
		signature A =
		 sig
			...
			datatype A = B |C |D
			val x  : int = ;
			val injection: int -> int ; 
		    
		 end
	\end{lstlisting}
	在签名中使用结构体的语法如下所示:
	\begin{lstlisting}
		sigature A =
			sig 
				...
				structure B :B
				...
			end
	\end{lstlisting}
	对于一个结构体,首先,这个结构体必须声明它实现的是哪个签名;其次,它必须实现对应签名所声明的所有函数;此外,结构体也可以声明签名以外的变量和函数,但是这些变量和函数不能被外界所访问。下面看一个结构体的例子:
	\begin{lstlisting}
		structure M :> B = (* B 对应要实现的签名*)
		struct
			structure R = R (*使用的结构体*)
			open R
			...函数实现省略
			
		end
	\end{lstlisting}	
	
	\section{词法分析简介}
	词法分析是程序处理正则表达式的第一步,主要的任务是将读入的字符串组成词素,并生成一个词法单元序列,然后将这个词法单元序列交给语法分析程序进行分析。\cite{appel2004modern} 由于正则表达式本身比较简单, 这个词法分析程序仅需要完成两个任务:第一个任务是规定词素的类型,并将词素的类型和输入的字符产序列进行模式匹配;第二个任务是识别出非法输入字符。在后面的章节中,我们可以看到,利用Standard ML语言的模式匹配特性,我们可以非常简明地实现这个词法分析程序。
	\section{递归下降语法制导分析简介}
	
	\subsection{上下文无关文法简介}
	上下文无关文法是语法分析重要的数学工具。这一节对上下文无关文法的介绍主要参考了\cite{2002编译原理}和\cite{Sips05a}, 其中Sipser\cite{Sips05a}对上下文无关文法的理论进行了完备的论述,但是高度形式化,而Aho\cite{2002编译原理}则详细地介绍了上下文无关文法在编译原理中的应用。
	
	首先,我们看一下上下文无关文法的定义:
	\begin{definition}[上下文无关文法]
		一个上下文无关文法是一个四元组\(\left( V, \Sigma , R , S \right) \) ,其中
		
		1. \(V\) 是一个有限集,称作非终结符
		
		2. \(\Sigma\) 是一个和\(V\) 相异的有限集,称作非终结符。
		
		3. \(R\) 是一个规则的有限集,其中每一个规则把一个非终结符替换为一个由终结符和非终结符所组成的序列,每一个规则被成为一个产生式。
		
		4. \( S \in V\) 是起始符号
		
	\end{definition}
	 其中,非终结符的层次结构可以用于正则表达式中不同操作符的优先级。而终结符应该是正则表达式中的操作符和普通字符。
	 
	 
	 \begin{definition}[上下文无关文的推导和句型]
	 	考虑上下文无关文法\(\left( V, \Sigma , R , S \right) \) ,假设\(A \rightarrow w\)是一个产生式, 对于一个序列\(\alpha A \beta\)根据上面的产生式替换为 \(\alpha w \beta\),也就是根据上面的产生式的一个推导。
	 	其中,如果一个序列可以通过起始符逐步推导得到,那么就称之为上下文无关文法的一个句型。
	 \end{definition}
 
 	下面这个概念对于递归下降语法制导分析非常有用。
 	\begin{definition}[最左推导]
 		如果每一步推导都选择每个句型的最左非终结符,那么这个推导过程就可以被称作最左推导。
 	\end{definition}
	\subsection{用上下文无关文法描述正则表达式}
	为了便于进行语法分析,我们
	基于定义2.5,我们可以使用上下文无关文法递归地描述正则表达式。但是在此之前,我们需要修改一下正则表达式约定的符号。由于计算机命令行程序难以处理"\(\left(\right)\)",并且没有"\(\cup\)"符号和\(\circ\),我们约定:\(\mbox{"}\left(\mbox{"和"}\right)\mbox{"}\)修改为\(\mbox{"}\left[  \mbox{"和"}\right]\mbox{"}\)。并且,我们用"\(+\)"代表"并"操作,即"\(\cup\)",用"\(r_1 r_2\)",也就是两个正则表达式直接连写来代表连结操作。
	
	现在,我们可以把定义3.5改写成如下的形式
	\begin{equation}
		r::= \quad \emptyset \quad| \quad\epsilon\quad |\quad a \quad  |\quad r_1 r_2\quad |\quad r_1 + r_2\quad | \quad r^*
	\end{equation}
	
	我们首先从优先级最低的并操作开始,我们约定,上下文无关文法的起始符是rexp,那么,我们可以得到:
	\begin{equation}
		rexp \quad::=\quad rterm\quad  | \quad rterm + rexp
	\end{equation}
	\indent 起始符的推导有两种情况,如果有并操作的话,这种情况可以分解为两个子表达式,另外一种情况是直接推导出下一个子表达式。对于在并操作下的情况,为了避免在后面的递归下降分析出现左递归的情况, 我们规定 "$+$"符号左边是新的非终结符。如果后续的表达式中还会出现"$+$"操作符,那继续展开右边的 $rexp$ 即可。\\
	\indent 对于优先级居中,而且也是双目操作的连结操作,我们可以像连结操作一样处理。这是因为:根据递归定义,我们可以把优先级更高、暂时不需要展开的正则表达式视作一个整体。在这种情况下,有更高等级的双目操作符可以抽象为同一个情况。在这种情况下,我们有:
	\begin{equation}
		rterm\quad::=\quad rfac\quad | \quad rfac\quad rterm
	\end{equation}
	\indent 对于优先级最高,而且是单目操作符的克莱因闭包,用上下文无关文法描述是比较容易的:
	\begin{equation}
		rfac\quad::=\quad ratom \quad | \quad ratom^* 
	\end{equation}
	\indent 根据递归定义,最低级的非终结$ratom$对应两种情况,第一种情况是直接推导出非终结符,第二种情况是推导出括号下的作为一个整体的正则表达式。
	\begin{equation}
		ratom\quad::=\quad @ \quad | \quad ! \quad |\quad a \quad | \quad[\quad rexp \quad]
	\end{equation}
	\indent 其中$@$对应的是空集, $!$ 对应的是空串, $a$对应的是字母表中任意一个字母, $\quad[\quad rexp \quad]$对应的是可能存在是括号内作为一个整体的正则表达式。 \\
	\indent 综上所述,我们得到了正则表达式的上下文无关文法的定义:
	\begin{equation}\label{cfg}
		\begin{split}
			rexp \quad::&=\quad rterm\quad  | \quad rterm + rexp \hfill  \\
			rterm\quad::&=\quad rfac\quad | \quad rfac\quad rterm \hfill  \\
			rfac\quad::&=\quad ratom \quad | \quad ratom^* \hfill \\
			ratom\quad::&=\quad @ \quad | \quad ! \quad |\quad a \quad | \quad[\quad rexp \quad] \hfill 
		\end{split}
	\end{equation}
	\indent 从\(\left( 3-6\right) \)中我们不难看出,每一个非终结符都能推导出下一个非终结符,而最后一个非终结符$ratom$可以推导出终结符,这样的话,我们可以说\ref{cfg}中的上下文无关文法是可达的。
	
	
%最后设置格式,插入参考文献。
\defaultfont
\bibliographystyle{gbt7714-2005}
\clearpage
\phantomsection
\addcontentsline{toc}{chapter}{参考文献}
\bibliography{document}
%插入致谢
\chapter*{致 \qquad 谢}
\addcontentsline{toc}{chapter}{致谢}
\thispagestyle{empty}
本科生涯看似漫长却又一晃而过,回首走过的岁月,我感慨良多。从最初的论文选题、思路梳理到研讨交流、反复修改直至最终完稿,都离不开老师、同学和亲人们的支持和无私帮助,在此我要向他们表达我最诚挚的谢意。
求学生涯暂告段落,但求知之路却永无止境。我将倍加珍惜大学生活给予我的珍贵财富,不忘初心,砥砺前行!
\chapter*{本科期间发表论文和科研情况}
\addcontentsline{toc}{chapter}{本科期间发表论文和科研情况}
\thispagestyle{empty}

...

\end{document}
