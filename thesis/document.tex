\documentclass[openany,oneside]{book}

\usepackage{jluthesisUTF8}
%\usepackage{gbt7714}
\usepackage{amsmath}
\usepackage{fontspec}
\usepackage{graphicx}
\usepackage{tikz}
\usepackage{multirow}
\usepackage{lipsum}
\usepackage{graphicx}
\usepackage{tabularx}
\usepackage{listings}
\usepackage{algorithmic}
\usepackage{algorithm}
\usepackage{pgf}
\bibliographystyle{gbt7714-2005}
%\usepackage{cite}


\makeatletter
\newenvironment{breakablealgorithm}
{% \begin{breakablealgorithm}
    \begin{center}
        \refstepcounter{algorithm}% New algorithm
        \hrule height.8pt depth0pt \kern2pt% \@fs@pre for \@fs@ruled
        \renewcommand{\caption}[2][\relax]{% Make a new \caption
            {\raggedright\textbf{\ALG@name~\thealgorithm} ##2\par}%
            \ifx\relax##1\relax % #1 is \relax
            \addcontentsline{loa}{algorithm}{\protect\numberline{\thealgorithm}##2}%
            \else % #1 is not \relax
            \addcontentsline{loa}{algorithm}{\protect\numberline{\thealgorithm}##1}%
            \fi
            \kern2pt\hrule\kern2pt
        }
    }{% \end{breakablealgorithm}
        \kern2pt\hrule\relax% \@fs@post for \@fs@ruled
    \end{center}
}
\makeatother

\usetikzlibrary{arrows}
\usepackage{xeCJK}

\floatname{algorithm}{算法}  
\renewcommand{\algorithmicrequire}{\textbf{输入:}}  
\renewcommand{\algorithmicensure}{\textbf{输出:}} 

%opening
\hypersetup{
    pdftitle    = {your pdf title},
    pdfsubject  = {your pdf subject},
    pdfkeywords = {your pdf keywords},
    pdfauthor   = {your name}
}



\begin{document}

\frontmatter
\sloppy % 解决中英文混排的断行问题,会加入间距,但不会影响断行 ????

%
% 手动在长标题中利用 \par 输入断行,
\ctitle{中文标题}
\etitle{English Title}
                                       % 论文 内容提要
\cthesissummary{
    中文摘要。
}
%                                           % 关键词
\ckeywords{甲, 乙, 丙}

\ethesissummary {
    Engligh abstract goes here.
}

\ekeywords{a, b, c}

\makecover


\pagenumbering{Roman} 
%\pdfbookmark[0]{目~~~~录}{contents}

\tableofcontents
{\xiaosi}
%{\fontsize \fontsize{12.05pt}{14.45pt}\selectfont}
% 清除目录后面空页的页眉和页脚
\clearpage{\pagestyle{empty}\cleardoublepage}

%%% 正文
\mainmatter
\defaultfont                        % 正文使用默认字体,小四,宋体

\chapter{绪论}
	正则表达式让我们拥有一种简单、具体而表达力强的语言来描述字符串中的各种模式。正则表达式因此可以被用作log 分析、 自然语言处理、 各种程序设计语言中的字符串匹配等需要处理字符串中结构化的数据的地方。正则表达式处理的高效性也允许我们去处理大规模的数据结合。标准的正则表达式匹配算法运行时间是 \(O\left( mn \right) \)
\section{研究背景及意义}
  	正则表达式
研究背景及意义,可用cite去引用。\cite{dean2008mapreduce}

\lipsum

\section{研究现状及挑战}

研究现状及挑战。

\section{研究内容与论文结构}

研究内容与论文结构。

\chapter{插入算法的示例}

可用如下方式插入算法流程:


\begin{breakablealgorithm}
    \caption{ACP 调度器}
    \label{algorithm: ACP}
        \begin{algorithmic}[1]
        \REQUIRE{要调度的作业集合$J$和资源集合$R$}
        \FOR{each r $\in$ R}
        %\STATE{test the c\_exec\_idx and the r\_exec\_idx of r with several benchmarking tools}
        \STATE{使用基准测试工具对$r$进行测试}
        \STATE{获取$r$的$storage_r$和$compute\_power_r$的值}
        \STATE{将$storage_r$和$compute\_power_r$写入到资源限制中}
        \ENDFOR
        \WHILE{true}
        \STATE{根据上次运行的结果,更新任务约束}
        \FOR{each r $\in$ R}
        \IF{r.schedular $\neq$ NULL}
        \IF{r.runningtask $\neq$ NULL}
        \STATE{rt $\leftarrow$ r.runningtask}
        \STATE{st $\leftarrow$ rt.starttime}
        \STATE{et $\leftarrow$ rt.endtime}
        \STATE{pd $\leftarrow$ rt.performancedemand}
        \STATE{sr $\leftarrow$ rt.scheduledresource}
        \STATE{根据(rt, st, et, pd, sr)更新约束条件}
        \STATE{$\slash\slash$ 根据任务的开始时间,结束时间,性能需求和其运行资源的性能添加一条约束}
        \ELSE
        \STATE{r.runningtask $\leftarrow$ NULL}
        \STATE{$\slash\slash$ 将该任务设置为已完成}
        \STATE{将 r.runningtask 从 r.schedular 中移除}
        \STATE{pj $\leftarrow$ r.parentjob}
        \IF{pj.schedular = NULL}
        \STATE{从$J$中删除pj}
        \ENDIF
        \ENDIF
        \ENDIF
        \ENDFOR
        \STATE{新建一个带有资源约束的模型}
        \STATE{将任务约束加入到OPL模型中}
        \STATE{将$J$和$R$作为约束添加到数据中}
        \STATE{求解OPL模型}
        \STATE{按照得出的最优解对任务进行分配}
        \STATE{等待新的事件出现}
        \ENDWHILE
    \end{algorithmic}
\end{breakablealgorithm}

\chapter{一级标题}

一级标题

\section{二级标题}

二级标题

\subsection{三级标题}

三级标题

%最后设置格式,插入参考文献。
\defaultfont
\bibliographystyle{gbt7714-2005}
\clearpage
\phantomsection
\addcontentsline{toc}{chapter}{参考文献}
\bibliography{document}
%插入致谢
\chapter*{致 \qquad 谢}
\addcontentsline{toc}{chapter}{致谢}
\thispagestyle{empty}
本科生涯看似漫长却又一晃而过,回首走过的岁月,我感慨良多。从最初的论文选题、思路梳理到研讨交流、反复修改直至最终完稿,都离不开老师、同学和亲人们的支持和无私帮助,在此我要向他们表达我最诚挚的谢意。

...

求学生涯暂告段落,但求知之路却永无止境。我将倍加珍惜大学生活给予我的珍贵财富,不忘初心,砥砺前行!
\chapter*{本科期间发表论文和科研情况}
\addcontentsline{toc}{chapter}{本科期间发表论文和科研情况}
\thispagestyle{empty}

...

\end{document}
